\documentclass[11pt]{article}

\usepackage[utf8]{inputenc}
\usepackage[spanish]{babel}

\usepackage{authblk}

\title{Simetría: Partículas como representaciones del grupo de Poincaré}
\author{Iván Mauricio Burbano Aldana}
\affil{Universidad de los Andes}

\begin{document}

\maketitle

\begin{abstract}
El desarrollo de la relatividad especial aclaró la estructura espaciotemporal sobre la cual se ha formulado el modelo estandar de partículas. En particular, resalta un conjunto de transformaciones, el grupo de Poincaré, que preserva esta estructura. Sorprendentemente, el requerimiento natural de que el conjunto de estados de un sistema cuántico sea un espacio de representación de este grupo, lleva de manera natural al concepto de partícula como representación irreducible etiquetada con masa y espín\cite{Wigner1939}. Ya que este requerimiento se puede extender a otros grupos de simetrías (internas), responsables de la diversidad de particulas en el modelo estandar, el ejemplo relativista forma un punto de partida importante para ejemplificar la interacción entre simetría y particulas. Despues de caracterizar el grupo de Poincaré siguiendo \cite{Scheck2010}, seguiremos el camino de ``físicos''\cite{Haag1992}\cite{Weinberg1995} para mostrar como el concepto de partícula surge de la invarianza de Lorentz. Sin embargo, se hará un esfuerzo por notar las partes no rigurosas de los argumentos estandar y, bajo la tutoría de \cite{Straumann2008}, indicar los caminos a solucionarlas.  
\end{abstract}

\bibliography{/home/ivan/Documents/Bib_Files/particle_physics_2017_II}
\bibliographystyle{ieeetr}

\end{document}